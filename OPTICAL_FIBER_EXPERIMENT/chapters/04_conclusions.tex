\chapter{Discussion and Conclusions}%
\begin{itemize}
    \item \textbf{Bending Loss Experiment}
    Macro-bending losses, on the other hand, occur in larger-radius bends and are primarily governed by geometric optics. When the incident angle of light exceeds the critical angle, it escapes the core and enters the cladding, resulting in losses. Our experiment demonstrated that the magnitude of macro-bending losses is influenced by factors such as the bend radius, core size, and the refractive indices of the core and cladding. As expected, smaller bend radii and larger core sizes led to increased losses.

    Moreover, we observed that as the bend radius decreased beyond a certain threshold, there was a sharp increase in bending losses. This critical bend radius is a crucial parameter to consider in practical applications, as it defines the fiber's flexibility limits while maintaining acceptable signal quality.
    \item \textbf{Calculation of Numerical Aperture}
    The NA is a fundamental parameter for characterizing optical fibers and plays a pivotal role in various optical systems, including telecommunications, medical imaging, and sensing applications.

    In our experiment, we determined the NA using the formula NA = n * $\sin{\theta}$, where "n" represents the refractive index of the core material, and $"\theta"$ is the half-angle of the maximum cone of light that can enter or exit the fiber while undergoing total internal reflection. We ensured precise measurements of "n" and "θ" to obtain accurate NA values, we got NA = $10.17^{o} \pm 0.001^{o}$.

    One notable finding is the direct correlation between the NA and the light-gathering ability of the optical fiber. Fibers with higher NAs can capture light from a wider range of angles, enabling them to transmit more light and information efficiently. 
    
    \item \textbf{Splice Loss Experiment}
    \begin{enumerate}
        \item \textbf{Loss Due to Tilt:} Tilt-induced splice loss occurs when the optical fibers are not perfectly aligned axially during the splicing process. Even a small angular misalignment can cause significant power loss due to light scattering at the splice interface. The experimental results show that splice loss due to tilt is directly proportional to the angle of tilt. This underscores the importance of precise alignment techniques during splicing. To minimize tilt-induced loss, it's essential to use specialized fusion splicing equipment with automated alignment capabilities or to carefully align the fibers manually.

         \item \textbf{Loss Due to Lateral Offset:} Lateral offset refers to the displacement of the fiber cores horizontally during the splicing process. Experimental data indicates that lateral offset leads to a notable splice loss, which increases with the magnitude of the offset. This effect is mainly caused by the mismatch between the fiber cores, resulting in imperfect coupling of light. To mitigate lateral offset-induced loss, precise fiber core alignment is crucial. Automated splicing machines or visual alignment aids can be employed to minimize lateral offset and enhance splice efficiency.

        \item \textbf{Loss Due to End Separation:} End separation, the axial gap between the fiber ends during splicing, is a critical factor contributing to splice loss. Experimental findings reveal that as the end separation increases, splice loss becomes more pronounced. This is attributed to the reduction in the overlap between the fiber cores, leading to inefficient light coupling. It is evident that maintaining minimal end separation is vital for reducing splice loss. Splicing equipment with precise end-face detection and adjustment mechanisms can help ensure optimal end separation and minimize signal attenuation.
    \end{enumerate}
    In this comprehensive experimental study, we delved into various critical aspects of optical fiber performance, encompassing bending losses, numerical aperture calculation, and splice losses attributed to tilt, lateral offset, and end separation. Our endeavor provided invaluable insights into the fundamental principles and practical implications governing optical fiber behavior.
    
Firstly, our investigation of bending losses elucidated the profound impact of curvature on signal transmission within optical fibers. We observed that both microbending and macrobending losses contribute significantly to signal attenuation. Microbending losses, caused by microscopic imperfections in the fiber core-cladding interface, underscored the importance of meticulous fiber handling during installation and maintenance. Meanwhile, macrobending losses, predominantly influenced by geometric optics, highlighted the relationship between bend radius, core size, and refractive indices. The identification of a critical bend radius served as a pivotal parameter for real-world applications, ensuring signal quality while accommodating bending requirements.

Secondly, our examination of numerical aperture (NA) calculations emphasized the NA's pivotal role in characterizing an optical fiber's light-gathering capacity. By applying the formula NA = n * $\sin(\theta)$, we determined that NA is contingent on both the refractive index (n) of the core and the maximum angle $(\theta)$ at which light can enter or exit the fiber while undergoing total internal reflection. This knowledge is indispensable for tailoring optical fibers to specific applications, such as data communication and medical imaging, where efficient light transmission and coupling are paramount.

Lastly, our analysis of splice losses due to tilt, lateral offset, and end separation underscored the significance of precise alignment during the splicing process. These losses occur when light cannot efficiently transition from one fiber to another due to misalignment. Our experiments revealed the substantial impact of even slight misalignments on signal attenuation. These findings underscore the importance of meticulous alignment procedures and high-quality splicing techniques to minimize signal degradation during the connection of optical fibers.

In summary, this multifaceted experiment has enriched our understanding of optical fiber physics and practical considerations. It equips us with the knowledge necessary for optimizing optical fiber networks, ensuring reliable and efficient data transmission across various applications. As we advance in the era of high-speed communication and data-intensive technologies, the insights gained from this study will continue to play a pivotal role in shaping the development and deployment of optical fiber systems.
\end{itemize}
