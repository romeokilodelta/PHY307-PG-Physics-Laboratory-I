\chapter{Discussion and Conclusions}%
The conducted experiment employing the ultrasonic interferometer for measuring ultrasonic velocity in a liquid and investigating the compressibility of the liquid has yielded significant insights into the acoustic properties of the studied medium. It was found that the ultrasonic velocity is \textbf{higher} in \textbf{denser liquid} and the liquid which is \textbf{less denser}, have lower ultrasonic velocity in it, has \textbf{greater} compressibility. 

From this we can understand, Sound is a mechanical wave and travels by compression and rarefaction of the medium. Its velocity in an elastic medium is proportional to the square root of Tension in the medium. A higher density leads to more elasticity in the medium and hence the ease by which compression and rarefaction can take place. This way the velocity of sound increases by increase in density.

Firstly, the ultrasonic interferometer, with its intricate setup involving a transducer crystal, fixed-frequency oscillator, and metal reflector, demonstrated its efficacy in generating and detecting ultrasonic waves within the liquid sample. By harnessing the principles of interference, stationary waves were created, allowing for the determination of nodal and antinodal positions.

Secondly, the precise measurement of the ultrasonic velocity in the liquid was achieved by varying parameters such as the distance between the transducer and the metal plate, and by observing corresponding changes in interference patterns. This velocity measurement is invaluable in understanding the medium's acoustic behavior and can serve as a basis for further studies in material characterization and fluid dynamics.

Furthermore, the experiment provided valuable data for calculating the compressibility of the liquid. Compressibility, a fundamental property indicating the medium's response to pressure changes, was accurately determined using the obtained ultrasonic velocity data and the bulk modulus equation. This information is essential for a wide array of applications, including the design and optimization of hydraulic systems and fluid-based technologies.


In conclusion, this experiment not only showcased the robustness and precision of ultrasonic interferometry in studying the acoustic properties of liquids but also highlighted its importance in scientific research and industrial applications.